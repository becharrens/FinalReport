% Here, you can define your own macros. Some examples are given below.

\newcommand{\R}[0]{\mathds{R}} % real numbers
\newcommand{\Z}[0]{\mathds{Z}} % integers
\newcommand{\N}[0]{\mathds{N}} % natural numbers
\newcommand{\C}[0]{\mathds{C}} % complex numbers
\renewcommand{\vec}[1]{{\boldsymbol{{#1}}}} % vector
\newcommand{\mat}[1]{{\boldsymbol{{#1}}}} % matrix


% {} limita scope de color
% Scribble syntax macros
\newcommand{\scrglobal}{{\color{blue}\mathtt{global}}}
\newcommand{\nested}{{\color{blue}\mathtt{nested}}}
\newcommand{\protocol}{{\color{blue}\mathtt{protocol}}}
\newcommand{\scrlocal}{{\color{blue}\mathtt{local}}}
\newcommand{\role}{{\color{blue}\mathtt{role}}}
\newcommand{\choice}{{\color{blue}\mathtt{choice}}}
\newcommand{\scrdo}{{\color{blue}\mathtt{do}}}
\newcommand{\rec}{{\color{blue}\mathtt{rec}}}
\newcommand{\calls}{{\color{blue}\mathtt{calls}}}
\newcommand{\continue}{{\color{blue}\mathtt{continue}}}
\newcommand{\from}{{\color{blue}\mathtt{from}}}
\newcommand{\scrto}{{\color{blue}\mathtt{to}}}
\newcommand{\scrend}{{\color{blue}\mathtt{end}}}
\newcommand{\scrnew}{{\color{blue}\mathtt{new}}}
\newcommand{\at}{{\color{blue}\mathtt{at}}}
\newcommand{\scror}{{\color{blue}\mathtt{or}}}
\newcommand{\invite}{{\color{blue}\mathtt{invite}}}
\newcommand{\create}{{\color{blue}\mathtt{create}}}
\newcommand{\scrin}{{\color{blue}\mathtt{in}}}
\newcommand{\accept}{{\color{blue}\mathtt{accept}}}

\newcommand{\project}{\projectenv{Env}}
\newcommand{\projectenv}[1]{\downarrow_A^{#1}}
\newcommand{\roles}[2]{\mathit{#1_1},\ ...\ ,\ \mathit{#1_#2}}
\newcommand{\rolessig}[2]{\role\ \mathit{#1_1},\ ...\ ,\ \role\ \mathit{#1_#2}}
\newcommand{\scrmessage}[2]{\mathtt{#1}(\mathtt{#2})}
\newcommand{\varinset}[2]{{#1 \in #2}}

\newcommand{\choiceG}[4]{\choice\ \at\ \mathit{#1}\ \{\,\mathit{#2}_{#3}\,\}_{\varinset{#3}{#4}}}
\newcommand{\messageG}[4]{\scrmessage{#1}{#2}\ \from\ \mathit{#3}\ \scrto\ \mathit{#4}}
\newcommand{\recG}[2]{\rec\ \mathit{#1}\ \{\,\mathit{#2}\,\}}
\newcommand{\continueG}[1]{\continue\ \mathit{#1}}
\newcommand{\callsG}[3]{\mathit{#1}\ \calls\ \mathit{#2}(\roles{#3}{n})}
\newcommand{\doG}[2]{\scrdo\ \mathit{#1}(\roles{#2}{n})}
\newcommand{\contG}[2]{#1;\ \mathit{#2}}

\newcommand{\choiceL}[4]{\choice\ \at\ \mathit{#1}\ \{\,\mathit{#2}_{#3}\,\}_{\varinset{#3}{#4}}}
\newcommand{\messageFromL}[3]{\scrmessage{#1}{#2}\ \from\ \mathit{#3}}
\newcommand{\messageToL}[3]{\scrmessage{#1}{#2}\ \scrto\ \mathit{#3}}
\newcommand{\recL}[2]{\rec\ \mathit{#1}\ \{\,\mathit{#2}\,\}}
\newcommand{\continueL}[1]{\continue\ \mathit{#1}}
\newcommand{\inviteL}[2]{\invite(\roles{#1}{n})\ \scrto\ \mathit{#2}}
\newcommand{\createL}[2]{\create(\rolessig{#1}{m})\ \scrin\ \mathit{#2}}
\newcommand{\acceptL}[4]{\accept\ \mathit{#1}(\roles{#2}{n};\ \scrnew\ \roles{#3}{m})\ \from\ \mathit{#4}}
\lstset{frame=tb,
  language=Java,
  aboveskip=3mm,
  belowskip=3mm,
  showstringspaces=false,
  columns=flexible,
  basicstyle={\ttfamily},
  numbers=left,
  numberstyle=\tiny\color{gray},
  keywordstyle=\color{blue},
  commentstyle=\color{dkgreen},
  stringstyle=\color{mauve},
  breaklines=true,
  breakatwhitespace=true,
  tabsize=3,
  frame=single,
}
\definecolor{dkgreen}{rgb}{0.078, 0.580, 0.086}
\definecolor{gray}{rgb}{0.5,0.5,0.5}
\definecolor{mauve}{rgb}{0.58,0,0.82}
\definecolor{granate}{rgb}{0.68, 0.02, 0.55}

\lstdefinelanguage{Scribble}{
  keywords={global, protocol, rec, role, choice, at, continue, do, from, to, var, if, in, while, do, else, case, break, calls, new, nested, accept, invite, create, local},
  keywordstyle=\color{blue}\bfseries,
  ndkeywords={class, export, boolean, throw, implements, import, this},
  ndkeywordstyle=\color{mauve}\bfseries,
  identifierstyle=\color{black},
  sensitive=false,
  comment=[l]{//},
  morecomment=[s]{/*}{*/},
  commentstyle=\color{dkgreen}\ttfamily,
  stringstyle=\color{dkgreen}\ttfamily,
  morestring=[b]',
  morestring=[b]"
}

\lstdefinelanguage{Pseudocode}{
  keywords={end, global, protocol, rec, role, choice, at, continue, do, from, to, var, if, in, while, do, else, case, break, calls, new, nested, accept, invite, create, local},
  % keywordstyle=\color{granate}\bfseries,
  ndkeywords={def, match, with, for, class, return, export, boolean, throw, implements, import, this},
  ndkeywordstyle=\color{mauve}\bfseries,
  identifierstyle=\color{black},
  sensitive=false,
  comment=[l]{//},
  morecomment=[s]{/*}{*/},
  commentstyle=\color{dkgreen}\ttfamily,
  stringstyle=\color{dkgreen}\ttfamily,
  morestring=[b]',
  morestring=[b]"
}


% NOTATION
\definecolor{purple}{rgb}{0.415, 0.035, 0.827}
% \definecolor{dkorange}{rgb}{0.870, 0.427, 0.007}
\definecolor{dkorange}{rgb}{0.709, 0.482, 0.050}
% \definecolor{dkred}{rgb}{0.878, 0.094, 0}
\definecolor{dkred}{rgb}{0.8, 0.019, 0}
% \definecolor{lightblue}{rgb}{0.047, 0.752, 0.811}
\definecolor{lightblue}{rgb}{0.043, 0.588, 0.6}
\definecolor{dkyellow}{rgb}{0.678, 0.654, 0}


% GO KWDS
\newcommand{\kwd}[1]{{\color{mauve} \texttt{#1}}}
\newcommand{\gotype}{\kwd{type}}
\newcommand{\gostruct}{\kwd{struct}}
\newcommand{\gointerface}{\kwd{interface}}
\newcommand{\gogo}{\kwd{go}}
\newcommand{\gofunc}{\kwd{func}}
\newcommand{\gochan}{\kwd{chan}}
% GO KWDS

% Reserved words
\newcommand{\protocolname}[1]{{\mathit{#1}}}
\newcommand{\rolename}[1]{{\mathit{#1}}}
\newcommand{\localprotocol}[2]{{#1@#2}}
\newcommand{\localprotocolname}[2]{{\mathit{#2}\_\mathit{#1}}}

% NOTATION
% Variables
\newcommand{\varname}[1]{{\mathtt{#1}}}

% Types
\newcommand{\typename}[1]{{\color{dkorange}\mathtt{#1}}}

% Directory structure
\newcommand{\filename}[1]{{\color{dkgreen} \mathtt{#1.go}}}
\newcommand{\dirname}[1]{{\color{dkgreen} \mathtt{#1}/}}
\newcommand{\pkgname}[1]{{\color{dkgreen} \mathtt{#1}}}

% Structs
\newcommand{\structname}[1]{{\color{dkred} \mathit{#1}}}
\newcommand{\structfield}[1]{{\color{dkred} \mathtt{#1}}}

% Functions and Interfaces
\newcommand{\functionname}[1]{\color{dkyellow} \mathtt{#1}}
\newcommand{\interfacename}[1]{\color{dkyellow} \mathit{#1}}

% Enums
\newcommand{\enumtype}[1]{\color{lightblue} \mathit{#1}}
\newcommand{\enumvalue}[1]{\color{lightblue} \mathtt{#1}}

% NOTATION

% HELPER CMDS
\newcommand{\payloaddecl}[3]{\structfield{#1_1}:\ \typename{#2_1},\ ...\ ,\ \structfield{#1_{#3}}:\ \typename{#2_{#3}}}
\newcommand{\pkgstructaccess}[2]{\pkgname{#1}.\structname{#2}}
% HELPER CMDS