%%%%%%%%%%%%%%%%%%%%%%%%%%%%%%%%%%%%%%%%%
% University Assignment Title Page 
% LaTeX Template
% Version 1.0 (27/12/12)
%
% This template has been downloaded from:
% http://www.LaTeXTemplates.com
%
% Original author:
% WikiBooks (http://en.wikibooks.org/wiki/LaTeX/Title_Creation)
%
% License:
% CC BY-NC-SA 3.0 (http://creativecommons.org/licenses/by-nc-sa/3.0/)
% 
%
%%%%%%%%%%%%%%%%%%%%%%%%%%%%%%%%%%%%%%%%%
%----------------------------------------------------------------------------------------
%	PACKAGES AND OTHER DOCUMENT CONFIGURATIONS
%----------------------------------------------------------------------------------------
\usepackage[a4paper,hmargin=2.8cm,vmargin=2.0cm,includeheadfoot]{geometry}
\usepackage{textpos}
\usepackage[numbers]{natbib} % for bibliography
\usepackage{tabularx,longtable,multirow,subfigure,caption}%hangcaption
\usepackage{fncylab} %formatting of labels
\usepackage{fancyhdr} % page layout
\usepackage{url} % URLs
\usepackage[english]{babel}
\usepackage{amsmath}
\usepackage{graphicx}
\usepackage{dsfont}
\usepackage{epstopdf} % automatically replace .eps with .pdf in graphics
% \usepackage{backref} % needed for citations
\usepackage{array}
\usepackage{latexsym}
\usepackage{semantic}
\usepackage{xfrac}
\usepackage{fourier}
\usepackage{ amssymb }


\usepackage{listings}
\usepackage{color}


\lstset{frame=tb,
  language=Java,
  aboveskip=3mm,
  belowskip=3mm,
  showstringspaces=false,
  columns=flexible,
  basicstyle={\ttfamily},
  numbers=left,
  numberstyle=\tiny\color{gray},
  keywordstyle=\color{blue},
  commentstyle=\color{dkgreen},
  stringstyle=\color{mauve},
  breaklines=true,
  breakatwhitespace=true,
  tabsize=3,
  frame=single,
}
\definecolor{dkgreen}{rgb}{0,0.6,0}
\definecolor{gray}{rgb}{0.5,0.5,0.5}
\definecolor{mauve}{rgb}{0.58,0,0.82}
\definecolor{granate}{rgb}{0.68, 0.02, 0.55}

\lstdefinelanguage{Scribble}{
  keywords={global, protocol, rec, role, choice, at, continue, do, from, to, var, if, in, while, do, else, case, break, calls, new, nested, accept, invite, create, local},
  keywordstyle=\color{granate}\bfseries,
  ndkeywords={class, export, boolean, throw, implements, import, this},
  ndkeywordstyle=\color{darkgray}\bfseries,
  identifierstyle=\color{black},
  sensitive=false,
  comment=[l]{//},
  morecomment=[s]{/*}{*/},
  commentstyle=\color{dkgreen}\ttfamily,
  stringstyle=\color{dkgreen}\ttfamily,
  morestring=[b]',
  morestring=[b]"
}


% \usepackage[pdftex,pagebackref,hypertexnames=false,colorlinks]{hyperref} % provide links in pdf
\usepackage[pdftex,hypertexnames=false,colorlinks]{hyperref} % provide links in pdf

\hypersetup{pdftitle={},
  pdfsubject={}, 
  pdfauthor={},
  pdfkeywords={}, 
  pdfstartview=FitH,
  pdfpagemode={UseOutlines},% None, FullScreen, UseOutlines
  bookmarksnumbered=true, bookmarksopen=true, colorlinks,
    citecolor=black,%
    filecolor=black,%
    linkcolor=black,%
    urlcolor=black}

\usepackage[all]{hypcap}

% {} limita scope de color
\newcommand{\scrglobal}{{\color{blue}\mathtt{global}}}
\newcommand{\nested}{{\color{blue}\mathtt{nested}}}
\newcommand{\protocol}{{\color{blue}\mathtt{protocol}}}
\newcommand{\scrlocal}{{\color{blue}\mathtt{local}}}
\newcommand{\role}{{\color{blue}\mathtt{role}}}
\newcommand{\choice}{{\color{blue}\mathtt{choice}}}
\newcommand{\scrdo}{{\color{blue}\mathtt{do}}}
\newcommand{\rec}{{\color{blue}\mathtt{rec}}}
\newcommand{\calls}{{\color{blue}\mathtt{calls}}}
\newcommand{\continue}{{\color{blue}\mathtt{continue}}}
\newcommand{\from}{{\color{blue}\mathtt{from}}}
\newcommand{\scrto}{{\color{blue}\mathtt{to}}}
\newcommand{\scrend}{{\color{blue}\mathtt{end}}}
\newcommand{\scrnew}{{\color{blue}\mathtt{new}}}
\newcommand{\at}{{\color{blue}\mathtt{at}}}
\newcommand{\scror}{{\color{blue}\mathtt{or}}}
\newcommand{\invite}{{\color{blue}\mathtt{invite}}}
\newcommand{\create}{{\color{blue}\mathtt{create}}}
\newcommand{\scrin}{{\color{blue}\mathtt{in}}}
\newcommand{\accept}{{\color{blue}\mathtt{accept}}}


%\usepackage{color}
%\usepackage[tight,ugly]{units}
%\usepackage{float}
%\usepackage{tcolorbox}
%\usepackage[colorinlistoftodos]{todonotes}
% \usepackage{ntheorem}
% \theoremstyle{break}
% \newtheorem{lemma}{Lemma}
% \newtheorem{theorem}{Theorem}
% \newtheorem{remark}{Remark}
% \newtheorem{definition}{Definition}
% \newtheorem{proof}{Proof}


%%% Default fonts
\renewcommand*{\rmdefault}{bch}
\renewcommand*{\ttdefault}{cmtt}



%%% Default settings (page layout)
\setlength{\parindent}{0em}  % indentation of paragraph

\setlength{\parindent}{0em}  % indentation of paragraph

\setlength{\headheight}{14.5pt}
\pagestyle{fancy}
\renewcommand{\chaptermark}[1]{\markboth{\chaptername\ \thechapter.\ #1}{}} 
%\fancyhead[RO]{\sffamily \textbf{\thepage}} %Page no.in the right on even pages
%\fancyhead[LE]{\sffamily \textbf{\thepage}} %Page no. in the left on odd pages

\fancyfoot[ER,OL]{\thepage}%Page no. in the left on
                                %odd pages and on right on even pages
\fancyfoot[OC,EC]{\sffamily }
\renewcommand{\headrulewidth}{0.1pt}
\renewcommand{\footrulewidth}{0.1pt}
% \captionsetup{margin=10pt,font=small,labelfont=bf}
\captionsetup{margin=10pt,font=small}

%--- chapter heading

\def\@makechapterhead#1{%
  \vspace*{10\p@}%
  {\parindent \z@ \raggedright \sffamily
    \interlinepenalty\@M
    \Huge\bfseries \thechapter \space\space #1\par\nobreak
    \vskip 30\p@
  }}

%--- chapter heading

\def\@makechapterhead#1{%
  \vspace*{10\p@}%
  {\parindent \z@ \raggedright \sffamily
        %{\Large \MakeUppercase{\@chapapp} \space \thechapter}
        %\\
        %\hrulefill
        %\par\nobreak
        %\vskip 10\p@
    \interlinepenalty\@M
    \Huge\bfseries \thechapter \space\space #1\par\nobreak
    \vskip 30\p@
  }}

%---chapter heading for \chapter*  
\def\@makeschapterhead#1{%
  \vspace*{10\p@}%
  {\parindent \z@ \raggedright
    \sffamily
    \interlinepenalty\@M
    \Huge \bfseries  #1\par\nobreak
    \vskip 30\p@
  }}	
\allowdisplaybreaks